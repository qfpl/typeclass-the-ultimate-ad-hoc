\documentclass[usenames,dvipsnames,svgnames,table,aspectratio=169,mathserif]{beamer}

\mode<presentation> {

%\usetheme{default}
\usetheme{Madrid}

\setbeamertemplate{footline} % To remove the footer line in all slides uncomment this line

\setbeamertemplate{navigation symbols}{} % To remove the navigation symbols from the bottom of all slides uncomment this line
}

\usepackage{graphicx} % Allows including images
\usepackage{booktabs} % Allows the use of \toprule, \midrule and \bottomrule in tables
\usepackage{hyperref}
\usepackage{apacite}
\usepackage{fancyvrb}
\usepackage{color}
\usepackage{alltt}
\usepackage{listings}
\usepackage{framed}
\usepackage{courier}
\usepackage{minted}
\usepackage{epstopdf}

\hypersetup{colorlinks=false}

\setbeamertemplate{bibliography entry title}{}
\setbeamertemplate{bibliography entry location}{}
\setbeamertemplate{bibliography entry note}{}
\setbeamertemplate{itemize items}[default]
\setbeamertemplate{enumerate items}[default]
\beamertemplatenavigationsymbolsempty
\setbeamertemplate{footline}{}

\newminted{haskell}{}
\newminted{java}{}
\newminted{scala}{}

\definecolor{g}{RGB}{0,100,0}
\newcommand{\highlight}[1]{\colorbox{yellow}{#1}}
\newcommand{\nega}[1]{\colorbox{yellow}{#1}}
\newcommand{\posi}[1]{\colorbox{green}{#1}}
\newcommand{\nl}{\vspace{\baselineskip}}
\newcommand{\pnl}{\pause \nl}

%%----------------------------------------------------------------------------------------
%	TITLE PAGE
%----------------------------------------------------------------------------------------

\title[Type Class: The Ultimate Ad Hoc]{Type Class: The Ultimate Ad Hoc} % The short title appears at the bottom of every slide, the full title is only on the title page
\titlegraphic{\includegraphics[scale=0.2]{data61.eps}}
\author{George Wilson} % Your name
\institute[] % Your institution as it will appear on the bottom of every slide, may be shorthand to save space
{
Data61/CSIRO\\ % Your institution for the title page
\medskip
\href{george.wilson@data61.csiro.au}{george.wilson@data61.csiro.au} % Your email address
}
\date{\today} % Date, can be changed to a custom date

\begin{document}


%%%%%
%%%%% Intro section
%%%%%


\begin{frame}
\titlepage % Print the title page as the first slide
\end{frame}


\begin{frame}

Type classes are a language feature

\begin{itemize}
\item Haskell
\item Purescript
\item Eta
\item Clean
\end{itemize}

or sometimes a design pattern

\begin{itemize}
\item Scala
\end{itemize}

\end{frame}


%%%%%
%%%%% Polymorphism
%%%%%

\begin{frame}
\begin{center}
\huge{Polymorphism}
\end{center}
\end{frame}


\begin{frame}

Something which is {\it polymorphic} has many shapes

\end{frame}


\begin{frame}
\begin{center}
Polymorphism is good


\begin{itemize}
\item less duplication
\item more reuse
\item many other benefits
\end{itemize}
\end{center}
\end{frame}


\begin{frame}
Broadly speaking there are two major forms of polymorphism in programming:

\begin{itemize}
\item {\it parametric} polymorphism
\item {\it ad-hoc} polymorphism
\end{itemize}
\end{frame}


\begin{frame}[fragile]

A {\it parametrically polymorphic} type has at least one {\it type parameter}
which can be instantiated to {\it any type}.

\nl

%Parametrically polymorphic functions behave the same way no matter which type
%they are instantiated to.

%\pnl

Example:

\begin{haskellcode}
reverse :: [a] -> [a]
\end{haskellcode}

\end{frame}


\begin{frame}[fragile]

An {\it ad-hocly polymorphic} type can be instantiated to some different types,

and may behave differently for each type

\nl

Example:

\begin{javacode}
==
\end{javacode}

\end{frame}


\begin{frame}
\begin{center}
\includegraphics[scale=0.5]{cromulent.png}
\end{center}
\end{frame}


\begin{frame}
\begin{center}
``[\ldots] {\it exhibits} ad-hoc polymorphism''
\end{center}
\end{frame}


\begin{frame}
\begin{center}
\includegraphics[scale=0.7]{programmar1.png}
\end{center}
\end{frame}


\begin{frame}
\begin{center}
\includegraphics[scale=0.8]{programmar2.png}
\end{center}
\end{frame}


%%%%%
%%%%% INTERFACES SECTION
%%%%%


\begin{frame}
\begin{center}
\huge{Interfaces}
\end{center}
\end{frame}


\begin{frame}[fragile]
\begin{javacode}
interface Equal<A> {
  public boolean eq(A other);
}
\end{javacode}

\pause

\begin{javacode}
class Person {
  public int age;
  public String name;




}
\end{javacode}
\end{frame}


\begin{frame}[fragile]
\begin{javacode}
interface Equal<A> {
  public boolean eq(A other);
}
\end{javacode}



\begin{javacode}
class Person implements Equal<Person> {
  public int age;
  public String name;

  public boolean eq(Person other) {
    return this.age == other.age && this.name.equals(other.name);
  }
}
\end{javacode}
\end{frame}


\begin{frame}[fragile]
\begin{javacode}
static <A extends Equal<A>> boolean elementOf(A a, List<A> list) {
  for (A element : list) {
    if (a.eq(element)) return true;
  }
  return false;
}
\end{javacode}

\end{frame}


\begin{frame}[fragile]
\begin{javacode}
                        "hello".eq("he11o")
\end{javacode}
\end{frame}


\begin{frame}[fragile]
\begin{javacode}
package java.lang;

class String {
  private char[] value;
  // other definitions
}
\end{javacode}
\end{frame}


\begin{frame}[fragile]
\begin{javacode}
package java.lang;

class String implements Equal<String> {
  private char[] value;
  // other definitions
}
\end{javacode}
\end{frame}


\begin{frame}
\begin{center}
\includegraphics[scale=0.5]{list.pdf}
\end{center}
\end{frame}


\begin{frame}[fragile]
\begin{javacode}
class List<A> {
  // implementation details





}
\end{javacode}
\end{frame}


\begin{frame}[fragile]
\begin{javacode}
class List<A> implements Equal<List<A>> {
  // implementation details

  public boolean eq(List<A> other) {
    // implementation...

  }
}
\end{javacode}
\end{frame}


\begin{frame}
\begin{center}
\includegraphics[scale=0.5]{listcompare1.pdf}
\end{center}
\end{frame}
\begin{frame}
\begin{center}
\includegraphics[scale=0.5]{listcompare2.pdf}
\end{center}
\end{frame}
\begin{frame}
\begin{center}
\includegraphics[scale=0.5]{listcompare3.pdf}
\end{center}
\end{frame}
\begin{frame}
\begin{center}
\includegraphics[scale=0.5]{listcompare4.pdf}
\end{center}
\end{frame}


\begin{frame}[fragile]
\begin{javacode}
class List<A> implements Equal<List<A>> {
  // implementation details

  public boolean eq(List<A> other) {
    // implementation...
    // ... but how do we compare A for equality?
  }
}
\end{javacode}
\end{frame}


\begin{frame}
\begin{itemize}
\item Interface implementation can't be conditional
\item We can only implement interfaces for types we control
\end{itemize}

\end{frame}


%%%%%
%%%%% WHAT ARE TYPE CLASSES
%%%%%


\begin{frame}
\begin{center}
\huge{Type Classes}
\end{center}
\end{frame}


\begin{frame}[fragile]
\begin{haskellcode}
class Equal a where
  eq :: a -> a -> Bool
\end{haskellcode}

\pnl

\begin{haskellcode}
data Person = Person {
  age :: Int
, name :: String
}
\end{haskellcode}

\pnl

\begin{haskellcode}
instance Equal Person where
  eq p1 p2 = eq (age p1) (age p2) && eq (name p1) (name p2)
\end{haskellcode}

\end{frame}


\begin{frame}[fragile]
\begin{haskellcode}
elementOf :: Equal a => a -> [a] -> Bool
elementOf a list =
  case list of
    []    -> False
    (h:t) -> eq a h || elementOf a t
\end{haskellcode}
\end{frame}


\begin{frame}[fragile]
Instances can be constrained

\nl

\begin{haskellcode}
instance (Equal a) => Equal [a] where
  eq []     []     = True
  eq (x:xs) []     = False
  eq []     (y:ys) = False
  eq (x:xs) (y:ys) = eq x y && eq xs ys
\end{haskellcode}

\pnl

We can add type class instances for types we didn't write
\end{frame}


\begin{frame}

Some benefits:
\begin{itemize}
\item You can write instances for types you did not write
\item Instances can depend on other instances
\end{itemize}

\nl

Compared to Interfaces:
\begin{itemize}
\item More expressive
\item More modular
\end{itemize}

\end{frame}


\begin{frame}
Type classes have restrictions in order to enforce {\it type class coherence}

\nl

Informally, coherence means:
\begin{itemize}
\item for a given type class for a given type, there is zero or one instance 
\item no matter how you ask for an instance, you get the same one
\item if an instance exists, you can't not get it
\end{itemize}
\end{frame}


\begin{frame}[fragile]

There are exactly two places a type class instance is allowed to exist

\nl

\begin{columns}
\begin{column}[T]{0.49\textwidth}
\begin{block}{\tt Person.hs}
\begin{haskellcode}
data Person = Person
  { age: Int
  , name: String }

instance Equal Person where
  eq p1 p2 = ...
\end{haskellcode}
\end{block}
\end{column}
\begin{column}{0.02\textwidth}
\end{column}
\begin{column}[T]{0.49\textwidth}
\begin{block}{\tt Equal.hs}
\begin{haskellcode}
class Equal a where
  eq :: a -> a -> Bool
\end{haskellcode}
\end{block}
\end{column}
\end{columns}

\end{frame}


\begin{frame}[fragile]

There are exactly two places a type class instance is allowed to exist

\nl

\begin{columns}
\begin{column}[T]{0.49\textwidth}
\begin{block}{\tt Person.hs}
\begin{haskellcode}
data Person = Person
  { age: Int
  , name: String }
\end{haskellcode}
\end{block}
\end{column}
\begin{column}{0.02\textwidth}
\end{column}
\begin{column}[T]{0.49\textwidth}
\begin{block}{\tt Equal.hs}
\begin{haskellcode}
class Equal a where
  eq :: a -> a -> Bool


instance Equal Person where
  eq p1 p2 = ...
\end{haskellcode}
\end{block}
\end{column}
\end{columns}

\end{frame}


\begin{frame}[fragile]

\begin{columns}
\begin{column}[T]{0.49\textwidth}
\begin{block}{\tt Person.hs}
\begin{haskellcode}
data Person = Person
  { age: Int
  , name: String }
\end{haskellcode}
\end{block}
\end{column}
\begin{column}{0.02\textwidth}
\end{column}
\begin{column}[T]{0.49\textwidth}
\begin{block}{\tt Equal.hs}
\begin{haskellcode}
class Equal a where
  eq :: a -> a -> Bool
\end{haskellcode}
\end{block}
\end{column}
\end{columns}

\begin{columns}
\begin{column}{0.5\textwidth}
\begin{block}{\tt EqualInstances.hs}
\begin{haskellcode}
instance Equal Person where
  eq p1 p2 = ...
\end{haskellcode}
\end{block}
\end{column}
\end{columns}
\pause
\begin{center}
``Orphan instance''

Orphan instances can break coherence
\end{center}

\end{frame}


\begin{frame}
Type class coherence benefits sanity:

\begin{itemize}
\item When you use a type class, the thing you expect happens
\item Instances never depends on imports or ordering
\item ``plumbing'' is done behind the scenes and can't go wrong
\end{itemize}

\pnl

Type class coherence rules out:

\begin{itemize}
\item Custom local instances
\item Multiple, selectable instances
\end{itemize}

(But there are other solutions to those things)
\end{frame}

%%%%%
%%%%% IMPLICITS (SCALA) SECTION
%%%%%


\begin{frame}
\begin{center}
\Huge{Implicits}

\nl

\large{{\it More Flexible Than Typeclasses}\texttrademark}
\end{center}
\end{frame}


\begin{frame}[fragile]
\begin{scalacode}
case class Person(age: Int, name: String)
\end{scalacode}

\pnl

\begin{scalacode}
trait Equal[A] {
  def eq(a: A, b: A): Boolean
}
\end{scalacode}

\pnl

\begin{scalacode}
implicit def equalPerson: Equal[Person] = new Equal[Person] {
  def eq(a: Person, b: Person): Boolean =
    a.age == b.age && a.name == b.name
}
\end{scalacode}
\end{frame}


\begin{frame}[fragile]
\begin{scalacode}
def elementOf[A](a: A, list: List[A])
                (implicit equalA: Equal[A]): Boolean = {
  list match {
    case Nil => false
    case (h::t) => equal.eq(a, h) || elementOf(a, t)
  }
}
\end{scalacode}
\end{frame}


\begin{frame}[fragile]
\begin{scalacode}
implicit def equalList(implicit equalA: Equal[A]): Equal[List[A]] =
  new Equal[List[A]] {
    def eq(a: List[A], b: List[A]): Boolean = {
      (a,b) match {
        case (Nil,   Nil)   => true
        case (x::xs, Nil)   => false
        case (Nil,   y::ys) => false
        case (x::xs, y::ys) => equalA.eq(x,y) || eq(xs,ys)
      }
    }
  }
\end{scalacode}
\end{frame}


\begin{frame}
\begin{itemize}
\item We can define implicits for types we did not write
\item We can write implicits that depend on implicits
\end{itemize}

\pnl

\begin{itemize}
\item No restriction on orphan instances
\item No restriction on number of instances
\end{itemize}
\end{frame}


\begin{frame}[fragile]
\begin{scalacode}
sealed trait Ordering
case object LT extends Ordering
case object EQ extends Ordering
case object GT extends Ordering
\end{scalacode}

\pnl

\begin{scalacode}
trait Order[A] {
  def compare(a: A, b: A): Ordering
}
\end{scalacode}

\pnl

\begin{scalacode}
implicit def joyDivisionWithoutIan = new Order[Person] {
  def compare(a: Person, b: Person): Ordering =
    intOrder.compare(a.age, b.age) match {
      case LT => LT
      case EQ => stringOrder.compare(a.name, b.name)
      case GT => GT
    }
}
\end{scalacode}
\end{frame}


\begin{frame}[fragile]
\begin{scalacode}
def sort[A](list: List[A])(implicit orderA: Order[A]): List[A] = {
  // quicksort goes here
}
\end{scalacode}
\end{frame}


\begin{frame}[fragile]
\begin{scalacode}
sort(
  List(
    Person(30, "Robert")
  , Person(20, "John")
  , Person(40, "Alfred")
  )
)
\end{scalacode}

\pnl

\begin{scalacode}
==>

List(
  Person(20, "John")
, Person(30, "Robert")
, Person(40, "Alfred")
)
\end{scalacode}
\end{frame}


\begin{frame}[fragile]
Then the boss says ``I want those sorted by name''.

\pnl

\begin{scalacode}
implicit def orderPersonByName: Order[Person] = new Order[Person] {
  def compare(a: Person, b: Person): Ordering =
    stringOrder.compare(a.name, b.name) match {
      case LT => LT
      case EQ => intOrder.compare(a.age, b.age)
      case GT => GT
    }
}
\end{scalacode}


\end{frame}


\begin{frame}[fragile]
\begin{scalacode}
sort(
  List(
    Person(30, "Robert")
  , Person(20, "John")
  , Person(40, "Alfred")
  )
)
\end{scalacode}

\pnl

\begin{scalacode}
==>

List(
  Person(40, "Alfred")
, Person(20, "John")
, Person(30, "Robert")
)

\end{scalacode}
\end{frame}


\begin{frame}[fragile]
\begin{scalacode}
// both in scope
implicit def orderPersonByAge: Order[Person] = ...
implicit def orderPersonByName: Order[Person] = ...

// what happens?
sort(persons)
\end{scalacode}

\pnl

Hopefully a compiler error!
\end{frame}


\begin{frame}
\begin{Huge}
\begin{center}
$\{1,2,3\} \cup \{4,5,6\}$
\end{center}
\end{Huge}
\end{frame}


\begin{frame}[fragile]
\begin{block}{Set.scala}
\begin{scalacode}
def emptySet[A]: Set[A]

def insert[A](a: A, set: Set[A])(implicit o: Order[A]): Set[A]

def union[A](s1: Set[A], s2: Set[A])(implicit o: Order[A]): Set[A]

def isElement[A](a: A, set: Set[A])(implicit o: Order[A]): Boolean
\end{scalacode}
\end{block}
\end{frame}


\begin{frame}[fragile]
\begin{block}{Persons.scala}
\begin{scalacode}
implicit def orderPersonByAge: Order[Person] = ...

def persons: Set[Person] =
  insert(p1, insert(p2, insert(p3, emptySet)))
\end{scalacode}
\end{block}

\nl

\begin{block}{Something.scala}
\begin{scalacode}
import Persons.{p1, persons}

implicit def orderPersonByName: Order[Person] = ...

val x = isElement(p1, persons) // FALSE!
\end{scalacode}
\end{block}
\end{frame}


\begin{frame}
Recommendations when writing implicits:
\begin{itemize}
\item Only create instances in the file that defines the type or the ``type class''
\item Disallow creating more than one instance (regardless of which file you're in)
\end{itemize}

\pnl

What about implicits in external libraries?
\begin{itemize}
\item Assess their usage of implicits. Do they use them as like type classes?
\item If you distrust their implicits, pass everything of theirs explicitly
\end{itemize}
\end{frame}


\begin{frame}[fragile]
\begin{center}
{\Huge \tt
:)
}
\end{center}
\end{frame}


\begin{frame}[fragile]
\small{
\begin{verbatim}
D.hs:11:10: error:
    Duplicate instance declarations:
      instance Ord Person -- Defined at D.hs:11:10
      instance Ord Person -- Defined at D.hs:14:10
\end{verbatim}
}
\end{frame}


\begin{frame}[fragile]
\scriptsize{
\begin{verbatim}
O3.hs:6:1: warning: [-Worphans]
    Orphan instance: instance Equal Person
    To avoid this
        move the instance declaration to the module of the class or of the type, or
        wrap the type with a newtype and declare the instance on the new type.

<no location info>: error: 
Failing due to -Werror.
\end{verbatim}
}
\end{frame}


\begin{frame}

Type classes:
\begin{itemize}
\item Big wins in flexibility, expressiveness, and modularity
\item Restrictions are straightforward and compiler checked
\item Coherence keeps things sane
\end{itemize}
\end{frame}


\begin{frame}

\begin{center}
{\Huge Thanks for listening!}
\end{center}

\begin{table}
\begin{tabular}{| l | c | c | c |}
\hline
Aspect                           & Interfaces & Type classes & Implicits \\
\hline
Instance types you control       & \checkmark & \checkmark & \checkmark     \\
Instance types you don't control & X          & \checkmark & \checkmark     \\
Instances can depend on other instances & X   & \checkmark & \checkmark     \\
Type-directed                    & \checkmark & \checkmark & sort of        \\
Custom local instances           & X          & X          & \checkmark     \\
Coherent                         & \checkmark & \checkmark & X              \\
\hline
\end{tabular}
\end{table}

\end{frame}


\end{document} 

